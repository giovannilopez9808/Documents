\documentclass[reprint,amsmath,amssymb,aps,]{revtex4-1}
\usepackage{graphicx}
\usepackage{dcolumn}
\usepackage{bm}
\usepackage{emptypage}
\usepackage{scrextend}
\pagestyle{empty}
\begin{document}
\title{Estudio de la irradiancia solar con dos modelos de transferencia radiativa y su comparación con mediciones en la ciudad de Monterrey}
\author{Gamaliel López-Padilla$^1$,Adriana Ipiña$^3$,Martin Freire $^{3,4}$,Beneddeto Schiavo$^{2}$,Rubén D. Piacentini$^{3,4}$}
\affiliation{1. Universidad Autónoma de Nuevo León\\
2. Centro de Ciencias de la Atmósfera, UNAM\\
3. Instituto de Física Rosario, CONICET-UNR\\
4. Facultad de Ciencias Exactas Ingeniería y Agrimensura, UNR}
\begin{abstract}
Determinar la intensidad solar que llega a nivel del suelo nos permite identificar y cuantificar los componentes que afectan en su atenuación. Presentamos un análisis de la irradiancia solar global (VIS-NIR*) en el periodo 2015-2018 medida por el Sistema Integral de Monitoreo Ambiental de Nuevo León, utilizada como referencia para la aproximación con los modelos TUV y SMARTS. Se seleccionaron mediciones bajo cielo despejado de la estación San Nicolás (25.75N, 100.26W, 512 m.s.n.m.) y se varió en los modelos la profundidad óptica de aerosol en 550nm entre 0.2-0.65, el albedo de dispersión simple entre 0.87-0.93 y el NO$_2$ superficial en el rango 0.1-1DU hasta lograr cercanía con la medición. Los resultados muestran que las diferencias relativas entre los valores obtenidos por ambos modelos son menores al 5\% al mediodía solar. Se discute la variación de las diferencias relativas en función de las horas del día debido a la influencia principalmente de agentes antropogénicos.
\end{abstract}
\maketitle
a
%\section{\label{sec:level1}Introducción}
%\section{\label{sec:level1}Objetivo}
%\section{\label{sec:level1}Hipótesis}
%\section{\label{sec:level1}Marco teórico}
%\section{\label{sec:level1}Materiales y métodos}
%\section{\label{sec:level1}Resultados}
%\section{\label{sec:level1}Discusión}
%\section{\label{sec:level1}Conclusiones}
%\section{\label{sec:level1}Referencias}
\end{document}